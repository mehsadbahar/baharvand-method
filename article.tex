\documentclass[12pt]{article}
\usepackage{amsmath}
\usepackage{amssymb}
\usepackage{booktabs}
\usepackage{geometry}
\usepackage{persian}
\usepackage{graphicx}

\settextfont{XB Zar}
\geometry{a4paper, margin=1in}

\title{مطالعه جامع روش کاهش اعداد: روش مهدی صادق بهاروند}
\author{مهدی صادق بهاروند}
\date{}

\begin{document}

\maketitle

\begin{abstract}
این مقاله به مطالعه روش کاهش اعداد مهدی صادق بهاروند می‌پردازد. این روش یک الگوریتم سیستماتیک برای کاهش اعداد طبیعی به صفر از طریق دنباله‌ای قطعی از تفریق‌ها است.
\end{abstract}

\section{مقدمه}
روش مهدی صادق بهاروند یک رویکرد سیستماتیک برای کاهش اعداد به صفر است.

\section{روش بهاروند}
برای یک عدد $N$ با $m$ رقم:
\begin{align*}
N_0 &= N \\
N_{i+1} &= N_i - 10^{p_i}
\end{align*}

\end{document}
