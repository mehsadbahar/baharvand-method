\documentclass[12pt]{article}
\usepackage{amsmath}
\usepackage{amssymb}
\usepackage{booktabs}
\usepackage{geometry}
\usepackage{persian}
\usepackage{graphicx}
\usepackage{algorithm}
\usepackage{algpseudocode}
\usepackage{listings}
\usepackage{xcolor}
\usepackage{float}
\usepackage{enumitem}
\usepackage{tikz}
\usepackage{pgfplots}
\pgfplotsset{compat=1.18}

\settextfont{XB Zar}
\geometry{a4paper, margin=1in}

\title{مطالعه جامع روش کاهش اعداد: روش مهدی صادق بهاروند}
\author{مهدی صادق بهاروند}
\date{}

\begin{document}

\maketitle

\begin{abstract}
این مقاله به مطالعه روش کاهش اعداد مهدی صادق بهاروند می‌پردازد. این روش یک الگوریتم سیستماتیک برای کاهش اعداد طبیعی به صفر از طریق دنباله‌ای قطعی از تفریق‌ها است. در این پژوهش، به تحلیل دقیق کاهش اعداد 12345 و 6789، بررسی دنباله اعداد متقارن و ارائه تحلیل‌های ریاضی پرداخته شده است.
\end{abstract}

\textbf{کلمات کلیدی:} کاهش اعداد، روش بهاروند، چرخه موقعیتی، اعداد متقارن، تحلیل ریاضی

\section{مقدمه}
روش مهدی صادق بهاروند یک رویکرد سیستماتیک برای کاهش اعداد به صفر است که بر اساس چرخه‌ای ثابت از موقعیت‌های ارقام عمل می‌کند. این روش نه تنها از نظر محاسباتی جالب توجه است، بلکه کاربردهای آموزشی ارزشمندی در درک مفاهیم پایه ریاضی دارد.

\section{روش بهاروند}

\subsection{فرمول‌بندی ریاضی}
برای یک عدد $N$ با $m$ رقم:
\begin{align*}
N_0 &= N \\
N_{i+1} &= N_i - 10^{p_i} \\
p_i &= (m - 1 - (i \mod m)) \\
m &= \lfloor \log_{10} N \rfloor + 1
\end{align*}

\section{نتایج}
تحلیل‌های انجام شده نشان می‌دهد که روش بهاروند برای اعداد متقارن بهینه عمل می‌کند.

\end{document}
